% Options for packages loaded elsewhere
% Options for packages loaded elsewhere
\PassOptionsToPackage{unicode}{hyperref}
\PassOptionsToPackage{hyphens}{url}
\PassOptionsToPackage{dvipsnames,svgnames,x11names}{xcolor}
%
\documentclass[
  letterpaper,
  DIV=11,
  numbers=noendperiod]{scrreprt}
\usepackage{xcolor}
\usepackage{amsmath,amssymb}
\setcounter{secnumdepth}{5}
\usepackage{iftex}
\ifPDFTeX
  \usepackage[T1]{fontenc}
  \usepackage[utf8]{inputenc}
  \usepackage{textcomp} % provide euro and other symbols
\else % if luatex or xetex
  \usepackage{unicode-math} % this also loads fontspec
  \defaultfontfeatures{Scale=MatchLowercase}
  \defaultfontfeatures[\rmfamily]{Ligatures=TeX,Scale=1}
\fi
\usepackage{lmodern}
\ifPDFTeX\else
  % xetex/luatex font selection
\fi
% Use upquote if available, for straight quotes in verbatim environments
\IfFileExists{upquote.sty}{\usepackage{upquote}}{}
\IfFileExists{microtype.sty}{% use microtype if available
  \usepackage[]{microtype}
  \UseMicrotypeSet[protrusion]{basicmath} % disable protrusion for tt fonts
}{}
\makeatletter
\@ifundefined{KOMAClassName}{% if non-KOMA class
  \IfFileExists{parskip.sty}{%
    \usepackage{parskip}
  }{% else
    \setlength{\parindent}{0pt}
    \setlength{\parskip}{6pt plus 2pt minus 1pt}}
}{% if KOMA class
  \KOMAoptions{parskip=half}}
\makeatother
% Make \paragraph and \subparagraph free-standing
\makeatletter
\ifx\paragraph\undefined\else
  \let\oldparagraph\paragraph
  \renewcommand{\paragraph}{
    \@ifstar
      \xxxParagraphStar
      \xxxParagraphNoStar
  }
  \newcommand{\xxxParagraphStar}[1]{\oldparagraph*{#1}\mbox{}}
  \newcommand{\xxxParagraphNoStar}[1]{\oldparagraph{#1}\mbox{}}
\fi
\ifx\subparagraph\undefined\else
  \let\oldsubparagraph\subparagraph
  \renewcommand{\subparagraph}{
    \@ifstar
      \xxxSubParagraphStar
      \xxxSubParagraphNoStar
  }
  \newcommand{\xxxSubParagraphStar}[1]{\oldsubparagraph*{#1}\mbox{}}
  \newcommand{\xxxSubParagraphNoStar}[1]{\oldsubparagraph{#1}\mbox{}}
\fi
\makeatother


\usepackage{longtable,booktabs,array}
\usepackage{calc} % for calculating minipage widths
% Correct order of tables after \paragraph or \subparagraph
\usepackage{etoolbox}
\makeatletter
\patchcmd\longtable{\par}{\if@noskipsec\mbox{}\fi\par}{}{}
\makeatother
% Allow footnotes in longtable head/foot
\IfFileExists{footnotehyper.sty}{\usepackage{footnotehyper}}{\usepackage{footnote}}
\makesavenoteenv{longtable}
\usepackage{graphicx}
\makeatletter
\newsavebox\pandoc@box
\newcommand*\pandocbounded[1]{% scales image to fit in text height/width
  \sbox\pandoc@box{#1}%
  \Gscale@div\@tempa{\textheight}{\dimexpr\ht\pandoc@box+\dp\pandoc@box\relax}%
  \Gscale@div\@tempb{\linewidth}{\wd\pandoc@box}%
  \ifdim\@tempb\p@<\@tempa\p@\let\@tempa\@tempb\fi% select the smaller of both
  \ifdim\@tempa\p@<\p@\scalebox{\@tempa}{\usebox\pandoc@box}%
  \else\usebox{\pandoc@box}%
  \fi%
}
% Set default figure placement to htbp
\def\fps@figure{htbp}
\makeatother





\setlength{\emergencystretch}{3em} % prevent overfull lines

\providecommand{\tightlist}{%
  \setlength{\itemsep}{0pt}\setlength{\parskip}{0pt}}



 


\KOMAoption{captions}{tableheading}
\makeatletter
\@ifpackageloaded{bookmark}{}{\usepackage{bookmark}}
\makeatother
\makeatletter
\@ifpackageloaded{caption}{}{\usepackage{caption}}
\AtBeginDocument{%
\ifdefined\contentsname
  \renewcommand*\contentsname{Table of contents}
\else
  \newcommand\contentsname{Table of contents}
\fi
\ifdefined\listfigurename
  \renewcommand*\listfigurename{List of Figures}
\else
  \newcommand\listfigurename{List of Figures}
\fi
\ifdefined\listtablename
  \renewcommand*\listtablename{List of Tables}
\else
  \newcommand\listtablename{List of Tables}
\fi
\ifdefined\figurename
  \renewcommand*\figurename{Figure}
\else
  \newcommand\figurename{Figure}
\fi
\ifdefined\tablename
  \renewcommand*\tablename{Table}
\else
  \newcommand\tablename{Table}
\fi
}
\@ifpackageloaded{float}{}{\usepackage{float}}
\floatstyle{ruled}
\@ifundefined{c@chapter}{\newfloat{codelisting}{h}{lop}}{\newfloat{codelisting}{h}{lop}[chapter]}
\floatname{codelisting}{Listing}
\newcommand*\listoflistings{\listof{codelisting}{List of Listings}}
\makeatother
\makeatletter
\makeatother
\makeatletter
\@ifpackageloaded{caption}{}{\usepackage{caption}}
\@ifpackageloaded{subcaption}{}{\usepackage{subcaption}}
\makeatother
\usepackage{bookmark}
\IfFileExists{xurl.sty}{\usepackage{xurl}}{} % add URL line breaks if available
\urlstyle{same}
\hypersetup{
  pdftitle={Jonathan Kenan Budianto},
  pdfauthor={13523139 Jonathan Kenan Budianto},
  colorlinks=true,
  linkcolor={blue},
  filecolor={Maroon},
  citecolor={Blue},
  urlcolor={Blue},
  pdfcreator={LaTeX via pandoc}}


\title{Jonathan Kenan Budianto}
\usepackage{etoolbox}
\makeatletter
\providecommand{\subtitle}[1]{% add subtitle to \maketitle
  \apptocmd{\@title}{\par {\large #1 \par}}{}{}
}
\makeatother
\subtitle{Portfolio Asesmen II-2100 KIPP}
\author{13523139 Jonathan Kenan Budianto}
\date{2025-10-28}
\begin{document}
\maketitle

\renewcommand*\contentsname{Table of contents}
{
\hypersetup{linkcolor=}
\setcounter{tocdepth}{2}
\tableofcontents
}

\bookmarksetup{startatroot}

\chapter*{Salam Kenal Semua!}\label{salam-kenal-semua}
\addcontentsline{toc}{chapter}{Salam Kenal Semua!}

\markboth{Salam Kenal Semua!}{Salam Kenal Semua!}

\begin{figure}[H]

{\centering \includegraphics[width=1\linewidth,height=\textheight,keepaspectratio]{images/oks.jpg}

}

\caption{About Me}

\end{figure}%

Halo! Senang sekali Anda mampir!

Perkenalkan, saya Jonathan Kenan Budianto, atau panggil saja Kenan. Saya
adalah seorang pemimpi berusia 19 tahun dari Jakarta yang sedang memulai
salah satu petualangan paling seru dalam hidup: menjadi seorang
mahasiswa. Bagi saya, setiap hari adalah sebuah halaman baru yang siap
diisi dengan pembelajaran, tawa, dan tentu saja, cerita-cerita tak
terduga!

Saya tumbuh dalam kehangatan keluarga besar, di mana pintu rumah selalu
terbuka dan meja makan tak pernah sepi. Kakek dan nenek saya adalah
pahlawan saya; mereka mengajarkan sebuah pelajaran sederhana namun
sangat mendalam: bahwa kebaikan tidak mengenal batas, dan setiap orang
yang kita temui adalah bagian dari keluarga besar kita sendiri. Filosofi
inilah yang menjadi kompas hidup saya, sebuah pengingat untuk selalu
menyebarkan kepedulian dan membangun jembatan, bukan tembok.

Ada sebuah percikan semangat di dalam diri saya yang selalu menyala
paling terang ketika saya bisa menjadi bagian dari sesuatu yang lebih
besar. Saya sangat percaya pada kekuatan kolaborasi dan pemberdayaan.
Melihat sebuah ide tumbuh menjadi kenyataan, menyaksikan orang-orang di
sekitar saya mencapai potensi mereka, dan merasakan energi dari sebuah
komunitas yang bergerak bersama---itulah yang membuat saya bersemangat!
Visi saya sederhana: menggunakan setiap ilmu dan karakter yang saya
miliki untuk ikut serta membangun masyarakat yang lebih baik, lebih
adil, dan lebih sinergis.

Di sini, di ruang digital kecil ini, saya ingin mengajak Anda untuk
melihat dunia melalui mata saya. Ini bukan sekadar portofolio; ini
adalah kumpulan fragmen perjalanan saya---dari momen-momen penuh tawa,
pelajaran berharga, hingga impian-impian besar yang sedang saya kejar.

Jadi, mari kita mulai petualangan ini. Selamat menjelajah, dan saya
harap Anda menemukan sesuatu yang menginspirasi di sini!

\bookmarksetup{startatroot}

\chapter{UTS-1 All About Me}\label{uts-1-all-about-me}

Siapakah saya? Pertanyaan itu terdengar sederhana, namun jawabannya
adalah mozaik dari ribuan momen dan perjumpaan. Bagi saya, Jonathan
Kenan Budianto, ``diri'' bukanlah sesuatu yang statis, melainkan sebuah
cerita yang terus ditulis. Halaman ini adalah upaya saya untuk
membagikan beberapa bab terpenting dari cerita itu.

\subsection{Akar yang Menumbuhkan: Pelajaran dari Rumah (dan
Dapur)}\label{akar-yang-menumbuhkan-pelajaran-dari-rumah-dan-dapur}

Saya percaya kita adalah cerminan dari cinta yang kita terima. Fondasi
cerita saya dibangun di rumah kakek dan nenek, sebuah tempat di mana
kepedulian adalah udara yang kami hirup setiap hari. Tentu saja, udara
itu juga sering kali beraroma masakan nenek yang legendaris, yang entah
bagaimana selalu berhasil membuat semua masalah terasa lebih kecil.

Ada dua pelajaran penting yang saya dapatkan di sana. Pertama, dari
nenek saya yang tidak akan membiarkan seorang kurir pergi tanpa segelas
teh hangat. ``Mereka juga sedang berjuang,'' ujarnya. Dari beliau, saya
belajar \textbf{empati} dalam tindakan nyata.

Pelajaran kedua, dari kakek saya, adalah humor. Suatu hari saya bertanya
mengapa beliau selalu tersenyum, bahkan saat membetulkan keran yang
bocor untuk kesekian kalinya. Beliau menjawab, ``Karena kalau cemberut,
kerannya tetap bocor, tapi wajah ganteng kakek jadi ikut bocor.'' Dari
beliau, saya belajar bahwa tawa adalah perekat terbaik dalam keluarga
dan cara terbaik menghadapi masalah-masalah kecil dalam hidup.

\subsection{Dari Ilmu ke Aksi: Kesenangan dalam
Memberdayakan}\label{dari-ilmu-ke-aksi-kesenangan-dalam-memberdayakan}

Ada sebuah kepuasan yang tak ternilai ketika ilmu yang kita pelajari
bisa bertransformasi menjadi aksi nyata. Saya menemukan percikan
kebahagiaan itu saat terlibat dalam berbagai kegiatan, di mana saya bisa
melihat langsung bagaimana sebuah ide dapat menumbuhkan harapan dan
bagaimana kolaborasi dapat menciptakan perubahan.

Bagi saya, ini adalah inti dari ``aktualisasi diri'': bukan tentang
menjadi yang terbaik, tetapi tentang memberikan yang terbaik dari diri
kita untuk lingkungan sekitar. Ini adalah proses tanpa akhir untuk
menjadi pribadi yang lebih kritis, peduli, dan terlibat dalam membentuk
masyarakat yang ideal.

\subsection{Visi ke Depan: Menjadi Penulis Cerita yang
Berdampak}\label{visi-ke-depan-menjadi-penulis-cerita-yang-berdampak}

Perjalanan ini masih panjang, dan saya tahu akan ada banyak bab baru
yang menanti. Visi saya ke depan sederhana: terus belajar, terus
bertumbuh, dan terus memanfaatkan setiap kesempatan untuk memberi dampak
positif---sambil berusaha mengingat nasihat kakek untuk tidak cemberut
pada ``keran-keran bocor'' dalam hidup.

Pada akhirnya, ``All About Me'' bukan hanya tentang siapa saya sekarang,
tetapi tentang siapa saya ingin menjadi: seseorang yang ceritanya dapat
membawa dampak positif bagi banyak orang.

\bookmarksetup{startatroot}

\chapter{UTS-2 Songs for You}\label{uts-2-songs-for-you}

Ada kenangan yang begitu kuat hingga warnanya tak akan pernah pudar,
bahkan dalam lembaran foto hitam putih sekalipun. Musik, bagi saya,
adalah mesin waktu yang bisa membawa kembali kehangatan dari kenangan
itu. Lagu ini bukan sekadar lagu; ini adalah ucapan terima kasih saya
yang paling tulus untuk fondasi hidup yang telah diberikan.

\begin{center}\rule{0.5\linewidth}{0.5pt}\end{center}

\subsection{``Monokrom'' oleh Tulus}\label{monokrom-oleh-tulus}

\emph{Sebuah surat cinta untuk kakek, nenek, dan semua kebaikan yang
mereka tanamkan.}

Setiap kali mendengar lagu ini, saya seperti diajak kembali ke rumah
masa kecil saya. Tulus bernyanyi tentang ``lembaran foto hitam putih'',
dan saya teringat akan semua pelajaran tanpa kata yang saya terima dari
kakek dan nenek. Mereka mengajarkan bahwa inti dari hidup bukanlah apa
yang kita miliki, melainkan seberapa besar kepedulian yang kita bagikan.

Lirik ``Denganmu, gelap ku tak pernah kelabu'' adalah kalimat yang
paling tepat untuk menggambarkan peran mereka. Mereka adalah cahaya yang
memastikan bahwa perjalanan saya selalu dipenuhi warna harapan dan
kebaikan. Lagu ini adalah perayaan untuk masa lampau yang tidak akan
pernah hilang ditelan waktu, sebuah pengingat abadi akan akar dari mana
saya bertumbuh.

\url{https://youtu.be/QqJ-Vp8mvbk?si=EwTS6Vq6e2jJM6FR}

\begin{quote}
\textbf{Lirik Lengkap ``Monokrom''}

Lembaran foto hitam putih

Aku coba ingat lagi warna bajumu kala itu

Kali pertama di hidupku

Manusia lain memelukku

Lembaran foto hitam putih

Aku coba ingat lagi wangi rumah di sore itu

Kue cokelat dan susu

Dan tiga bocah di selebar koran sore

Di mana pun kalian berada

Kukirimkan terima kasih

Untuk warna dalam hidupku dan banyak kenangan indah

Kau melukis aku

Lembaran foto hitam putih

Kembali teringat malam kuhitung-hitung bintang

Saat mataku sulit tidur

Suaramu buatku lelap

Di mana pun kalian berada

Kukirimkan terima kasih

Untuk warna dalam hidupku dan banyak kenangan indah

Kau melukis aku

Kita tak pernah tahu berapa lama kita diberi waktu

Jika aku pergi lebih dulu, jangan lupakan aku

Ini lagu untukmu, ungkapan terima kasihku

Lembar monokrom hitam putih

Aku coba ingat warna demi warna di hidupku

Tak akan ku mengenal cinta

Bila bukan karena hati baikmu
\end{quote}

\bookmarksetup{startatroot}

\chapter{UTS-3 My Stories for You}\label{uts-3-my-stories-for-you}

Terkadang, pelajaran hidup yang paling berharga tidak datang dari
buku-buku tebal, melainkan dari momen-momen kecil yang tak terduga.
Cerita berikut ini adalah salah satu momen itu---sebuah pengingat
tentang dari mana saya berasal dan untuk apa saya berjuang.

\begin{center}\rule{0.5\linewidth}{0.5pt}\end{center}

\subsection{Pelajaran dari Kotak
Kardus}\label{pelajaran-dari-kotak-kardus}

Adik perempuan saya pulang dari sekolah dengan wajah murung. Di
tangannya ada secarik kertas tugas: membuat diorama ekosistem laut dari
bahan bekas. Bagi sebagian anak, ini mungkin tugas yang menyenangkan.
Tapi bagi adik saya yang saat itu merasa kurang percaya diri dengan
kemampuan seninya, tugas ini terasa seperti gunung yang mustahil didaki.

``Aku nggak bisa, Kak,'' keluhnya sambil meletakkan beberapa botol
plastik dan sebuah kotak kardus bekas di lantai. ``Pasti punya
teman-teman yang lain lebih bagus.''

Melihat keputusasaan di matanya, saya teringat akan pelajaran dari kakek
dan nenek: kepedulian terbesar sering kali hadir dalam bentuk waktu dan
perhatian. Saya pun duduk di sampingnya di lantai, di antara tumpukan
``sampah'' yang akan kami sulap. Saya tidak mengambil alih tugasnya.
Sebaliknya, saya hanya bertanya, ``Menurutmu, karang paling bagus dibuat
dari apa? Kalau ikan, bagaimana caranya agar bisa `berenang'?''

Selama tiga jam berikutnya, kami bekerja bersama. Saya membantunya
memotong bagian yang sulit, memberinya ide untuk membuat gurita dari
sisa benang wol, dan yang terpenting, meyakinkannya bahwa setiap
potongan yang ia tempel sudah sangat bagus. Perlahan, wajah murungnya
berubah menjadi senyum konsentrasi, lalu tawa bangga saat diorama itu
mulai terbentuk. Kotak kardus itu bukan lagi sekadar kotak kardus; ia
telah menjadi samudra penuh warna di ruang keluarga kami.

Esoknya, ia pulang dengan senyum yang jauh lebih lebar. Dioramanya
mendapat pujian dari guru. Tapi bukan itu kemenangannya. Kemenangan
sesungguhnya adalah kilau di matanya saat ia berkata, ``Kak, ternyata
aku bisa, ya?''

Momen itu mengajarkan saya sebuah pelajaran yang mendalam. Kebahagiaan
terbesar bukanlah saat kita mencapai sesuatu untuk diri kita sendiri,
tetapi saat kita bisa menjadi `percikan api' yang membantu orang lain
menemukan kekuatan dalam diri mereka. Pelajaran dari kotak kardus itu
terus saya bawa hingga hari ini, sebagai pengingat bahwa memberdayakan
orang lain adalah aktualisasi diri yang paling sejati.

\bookmarksetup{startatroot}

\chapter{UTS-4 My SHAPE}\label{uts-4-my-shape}

Memahami diri adalah langkah pertama untuk bisa memberi dampak. Analisis
SHAPE ini adalah proses saya memetakan dan menyatukan berbagai kepingan
yang membentuk identitas saya. Ini bukan sekadar daftar, melainkan
sebuah kompas yang saya susun untuk menavigasi perjalanan hidup,
memastikan setiap langkah selaras dengan panggilan hati saya.

\begin{center}\rule{0.5\linewidth}{0.5pt}\end{center}

\subsection{Peta SHAPE Saya}\label{peta-shape-saya}

\begin{itemize}
\tightlist
\item
  \textbf{S -- Strengths (Kekuatan Khas):}

  \begin{itemize}
  \tightlist
  \item
    \textbf{Empati:} Kemampuan untuk merasakan dan memahami perspektif
    orang lain, sebuah kekuatan yang ditanamkan sejak kecil oleh
    keluarga.
  \item
    \textbf{Pengembangan Orang Lain:} Menemukan energi dan kepuasan saat
    membantu orang lain bertumbuh dan menemukan potensi mereka.
  \item
    \textbf{Pemikiran Kritis \& Reflektif:} Kemampuan untuk menganalisis
    keadaan dan secara sadar merenungkan proses pengembangan diri untuk
    menjadi lebih baik.
  \item
    \textbf{Kepemimpinan Melayani:} Memimpin dengan memberi dukungan dan
    menciptakan lingkungan di mana setiap orang merasa dihargai dan bisa
    berkontribusi.
  \end{itemize}
\item
  \textbf{H -- Heart (Panggilan Hati):}

  \begin{itemize}
  \tightlist
  \item
    \textbf{Pemberdayaan Komunitas:} Hati saya terpanggil untuk terlibat
    dalam upaya menjadikan lingkungan sekitar menjadi lebih baik,
    sinergis, dan ideal.
  \item
    \textbf{Keadilan \& Kepedulian:} Sebuah hasrat mendalam untuk
    memastikan setiap individu, terlepas dari latar belakangnya,
    diperlakukan dengan baik dan menjadi bagian dari ``keluarga''.
  \item
    \textbf{Pertumbuhan Bersama:} Saya sangat percaya pada kekuatan
    kolaborasi dan sinergi; kita tumbuh paling kuat saat kita tumbuh
    bersama.
  \end{itemize}
\item
  \textbf{A -- Aptitudes \& Acquired Skills (Bakat \& Keterampilan):}

  \begin{itemize}
  \tightlist
  \item
    \textbf{Komunikasi Interpersonal:} Terlatih untuk mendengarkan dan
    membangun hubungan yang tulus, berkat lingkungan keluarga yang
    sangat komunal.
  \item
    \textbf{Manajemen Organisasi:} Memiliki kemampuan untuk mengelola
    kegiatan dan memberdayakan tim untuk mencapai tujuan bersama.
  \item
    \textbf{Fasilitasi dan Mentoring:} Mampu memandu dan mendukung orang
    lain dalam proses belajar dan pengembangan diri mereka.
  \end{itemize}
\item
  \textbf{P -- Personality (Gaya Kepribadian):}

  \begin{itemize}
  \tightlist
  \item
    \textbf{Kolaboratif:} Saya bekerja paling efektif saat berada dalam
    tim, berbagi ide, dan membangun sesuatu bersama-sama.
  \item
    \textbf{Berorientasi pada Orang Lain (People-Oriented):} Fokus saya
    secara alami tertuju pada kesejahteraan dan pertumbuhan orang-orang
    di sekitar saya.
  \item
    \textbf{Reflektif \& Introspektif:} Saya cenderung mengambil waktu
    untuk merenung dan memahami ``mengapa'' di balik setiap tindakan dan
    tujuan.
  \end{itemize}
\item
  \textbf{E -- Experiences (Pengalaman Kunci):}

  \begin{itemize}
  \tightlist
  \item
    \textbf{Pelajaran dari Keluarga Besar:} Dibesarkan di lingkungan
    yang mengutamakan kebersamaan dan kepedulian telah menjadi
    pengalaman formatif yang membentuk seluruh pandangan dunia saya
    tentang hubungan antarmanusia.
  \item
    \textbf{Terlibat dalam Pemberdayaan Lingkungan:} Setiap momen di
    mana saya bisa berkontribusi dalam kegiatan yang memberdayakan
    komunitas telah mengkonfirmasi panggilan hati saya dan memberikan
    kepuasan yang mendalam.
  \end{itemize}
\end{itemize}

\begin{center}\rule{0.5\linewidth}{0.5pt}\end{center}

\subsection{Piagam Diri (Self-Charter)}\label{piagam-diri-self-charter}

\textbf{Misi Hidup Saya:} \textgreater{} Menjadi katalisator kebaikan,
menggunakan empati dan semangat kolaborasi untuk membangun komunitas di
mana setiap individu merasa berdaya, dihargai, dan terinspirasi untuk
bertumbuh.

\textbf{Nilai Inti Saya:} \textgreater{} Empati, Kepedulian, Kolaborasi,
Pertumbuhan, dan Integritas.

\textbf{Janji Pelayanan Saya:} \textgreater{} Saya berjanji untuk selalu
hadir dengan telinga yang mendengar, hati yang peduli, dan tangan yang
siap membantu, menciptakan ruang yang aman bagi orang lain untuk menjadi
versi terbaik dari diri mereka.

\begin{center}\rule{0.5\linewidth}{0.5pt}\end{center}

\subsection{Narasi 90 Detik (Elevator
Pitch)}\label{narasi-90-detik-elevator-pitch}

``Halo, saya Kenan. Saya adalah seorang pembelajar dan pemberdaya yang
percaya bahwa perubahan terbesar dimulai dari kepedulian tulus.
Dibesarkan dalam keluarga yang mengajarkan bahwa semua orang adalah
bagian dari kita, saya membawa nilai empati dalam segala hal. Kekuatan
saya terletak pada kemampuan untuk menghubungkan orang dan membantu
mereka bertumbuh bersama. Hati saya terpanggil untuk membangun komunitas
yang sinergis dan ideal. Dengan pengalaman dalam memfasilitasi dan
berorganisasi, misi saya sederhana: mencapai aktualisasi diri dengan
cara memberdayakan orang-orang di sekitar saya.''

\bookmarksetup{startatroot}

\chapter{UTS-5 My Personal Reviews}\label{uts-5-my-personal-reviews}

Berikut cara saya melakukan review: mengguan chatGPT, saya mengattach
\href{skor_uts.pdf}{file promt ChatGPT}, disertai perintah :``self
assess uts-1 sanpai uts-5 dari URL
`https://ii-2100.github.io/all-about-me/'\,''

ChatGPT melakukan self-assessment UTS-1 s.d. UTS-5 langsung dari laman
yang Anda berikan dan menilai memakai rubrik tugas UTS (skala 1--5 per
kriteria). Rekap skor siap diunduh sebagai CSV:
\href{sandbox:/mnt/data/UTS_self_assessment.csv}{Download CSV
ringkasan}.

\bookmarksetup{startatroot}

\chapter{Hasil Self-Assessment UTS (URL:
ii-2100.github.io/all-about-me)}\label{hasil-self-assessment-uts-url-ii-2100.github.ioall-about-me}

\section{Identifikasi}\label{identifikasi}

\begin{itemize}
\tightlist
\item
  Nama \& NIM penulis: \textbf{Armein Z. R. Langi -- 131902360} (tertera
  di halaman depan portofolio).
  (\href{https://ii-2100.github.io/all-about-me/}{II 2100})
\item
  Penilai: \textbf{Self-assessment (Armein Z. R. Langi)}
\item
  Catatan cakupan: halaman beranda memuat ``About Me''; navigasi ke ``My
  Songs for You'', ``My Stories for You'', ``My Shapes'', dan ``My
  Personal Reviews'' tersedia.
  (\href{https://ii-2100.github.io/all-about-me/}{II 2100})
\end{itemize}

\section{Tinjauan Umum}\label{tinjauan-umum}

\begin{itemize}
\tightlist
\item
  \textbf{UTS-1 (All About Me)} hadir di beranda (``Selamat Berjumpa /
  About Me''). Isi memperkenalkan identitas dan latar personal secara
  padat. (\href{https://ii-2100.github.io/all-about-me/}{II 2100})
\item
  \textbf{UTS-2 (My Songs for You)} memuat judul karya dan tautan audio,
  namun lirik/isi tidak ditampilkan di halaman (file audio tidak bisa
  saya akses dari sini), sehingga penilaian konten terbatas pada
  kelengkapan presentasi.
  (\href{https://ii-2100.github.io/all-about-me/My_Song_for_You/index.html}{II
  2100})
\item
  \textbf{UTS-3 (My Stories for You)} berisi tautan ke beberapa cerita;
  konten naratif kuat, terutama pada kisah keluarga ``Gaun Pengantin
  Gladys'' serta kisah inspiratif lainnya.
  (\href{https://ii-2100.github.io/all-about-me/My_Stories_for_You/index.html}{II
  2100})
\item
  \textbf{UTS-4 (My SHAPE)} saat ini halaman kosong/tanpa isi, hanya
  navigasi.
  (\href{https://ii-2100.github.io/all-about-me/My_Shapes/index.html}{II
  2100})
\item
  \textbf{UTS-5 (My Personal Reviews)} berisi metode/tautan panduan
  review, namun belum ada contoh \textbf{review personal} yang lengkap
  terhadap sebuah pesan/teks.
  (\href{https://ii-2100.github.io/all-about-me/My_Personal_Reviews/index.html}{II
  2100})
\end{itemize}

\begin{center}\rule{0.5\linewidth}{0.5pt}\end{center}

\section{Tinjauan Spesifik + Skor
(1--5)}\label{tinjauan-spesifik-skor-15}

\subsection{UTS-1 --- All About Me (di
beranda)}\label{uts-1-all-about-me-di-beranda}

\textbf{Skor per kriteria:} Orisinalitas \textbf{3}, Keterlibatan
\textbf{3}, Humor \textbf{1}, Wawasan/Insight \textbf{3} → \textbf{Total
10/20 (50\%)}. \textbf{Alasan singkat:} Perkenalan diri faktual dan
jelas, namun gaya masih informatif ketimbang naratif (minim humor dan
moment ``hook''). (\href{https://ii-2100.github.io/all-about-me/}{II
2100}) \textbf{Saran perbaikan:} Buka dengan anekdot pribadi (1--2
paragraf) yang ``mengikat'' (mis. titik balik karier/keluarga),
tambahkan satu momen humor ringan, lalu tutup dengan refleksi singkat
tentang nilai/visi diri agar aspek insight naik.

\subsection{UTS-2 --- My Songs for You}\label{uts-2-my-songs-for-you}

\textbf{Skor per kriteria:} Orisinalitas \textbf{2}, Keterlibatan
\textbf{2}, Humor \textbf{1}, Inspirasi \textbf{2} → \textbf{Total 7/20
(35\%)}. \textbf{Alasan singkat:} Halaman menampilkan judul lagu ``River
in my Mind'', ``Heaven on Earth'', namun tanpa lirik/cerita di balik
lagu sehingga sulit menilai aspek pesan, humor, dan inspirasi.
(\href{https://ii-2100.github.io/all-about-me/My_Song_for_You/index.html}{II
2100}) \textbf{Saran perbaikan:} Tambahkan lirik lengkap, 1 paragraf
cerita proses kreatif, dan 2--3 kalimat ``pesan untukmu'' agar inspirasi
terbaca; sertakan player/tautan yang dapat diputar langsung + fallback
transkrip.

\subsection{UTS-3 --- My Stories for
You}\label{uts-3-my-stories-for-you-1}

\textbf{Skor per kriteria:} Orisinalitas \textbf{5}, Keterlibatan
\textbf{5}, Pengembangan Narasi \textbf{4}, Inspirasi \textbf{5} →
\textbf{Total 19/20 (95\%)}. \textbf{Alasan singkat:} Cerita ``Gaun
Pengantin Gladys'' dkk sangat personal, emosional, dan inspiratif; ritme
bertutur hidup serta detail situasional kuat (konten ramu unsur
penebusan/keluarga/iman).
(\href{https://ii-2100.github.io/all-about-me/My_Stories_for_You/index.html}{II
2100}) \textbf{Saran perbaikan:} Tambah ``lead'' 2--3 kalimat yang
merangkum pesan kunci tiap cerita; akhiri dengan ajakan/refleksi 1--2
kalimat agar resonansi ke pembaca makin jelas.

\subsection{UTS-4 --- My SHAPE}\label{uts-4-my-shape-1}

\textbf{Skor per kriteria:} Orisinalitas \textbf{1}, Keterlibatan
\textbf{1}, Pengembangan \textbf{1}, Inspirasi \textbf{1} →
\textbf{Total 4/20 (20\%)}. \textbf{Alasan singkat:} Konten belum
tersedia.
(\href{https://ii-2100.github.io/all-about-me/My_Shapes/index.html}{II
2100}) \textbf{Saran perbaikan (prioritas):}

\begin{enumerate}
\def\labelenumi{\arabic{enumi}.}
\tightlist
\item
  Tampilkan rekap SHAPE (Spiritual gifts, Heart, Abilities, Personality,
  Experiences) dalam 1 tabel ringkas + 3--5 butir bukti;
\item
  Narasikan 2 pengalaman kunci (±150 kata/each) yang menunjukkan
  ``signature strengths'';
\item
  Tutup dengan 3 rencana aksi konkret (who/what/when).
\end{enumerate}

\subsection{UTS-5 --- My Personal
Reviews}\label{uts-5-my-personal-reviews-1}

\textbf{Skor per kriteria:} Pemahaman Konsep \textbf{2}, Analisis Kritis
\textbf{1}, Argumentasi (Logos) \textbf{1}, Etos \& Empati \textbf{2},
Rekomendasi \textbf{1} → \textbf{Total 7/25 (28\%)}. \textbf{Alasan
singkat:} Halaman berisi metode cara menilai, tetapi \textbf{belum ada}
contoh \textbf{review personal} yang lengkap terhadap sebuah pesan/teks
sehingga aspek analisis-argumentasi tak terbaca.
(\href{https://ii-2100.github.io/all-about-me/My_Personal_Reviews/index.html}{II
2100}) \textbf{Saran perbaikan:} Pilih 1 karya personal (mis.
UTS-1/2/3), tulis review 400--600 kata: ringkas pesan, nilai dengan
rubrik, berikan 2--3 bukti kutipan, evaluasi etos/empati, lalu tutup
dengan rekomendasi perbaikan spesifik.

\begin{center}\rule{0.5\linewidth}{0.5pt}\end{center}

\section{Rekap Skor (ringkas)}\label{rekap-skor-ringkas}

\begin{itemize}
\tightlist
\item
  \textbf{UTS-1:} 10/20 → \textbf{50\%}
\item
  \textbf{UTS-2:} 7/20 → \textbf{35\%}
\item
  \textbf{UTS-3:} 19/20 → \textbf{95\%}
\item
  \textbf{UTS-4:} 4/20 → \textbf{20\%}
\item
  \textbf{UTS-5:} 7/25 → \textbf{28\%}
\end{itemize}

CSV lengkap sudah saya siapkan untuk dokumentasi dan olah lanjut:
\href{sandbox:/mnt/data/UTS_self_assessment.csv}{Download CSV
ringkasan}.

\section{Langkah Perbaikan Cepat (prioritas 1
minggu)}\label{langkah-perbaikan-cepat-prioritas-1-minggu}

\begin{enumerate}
\def\labelenumi{\arabic{enumi}.}
\tightlist
\item
  \textbf{Lengkapi UTS-4 (My SHAPE)} sesuai saran di atas;
\item
  \textbf{Tambahkan lirik + sinopsis} untuk tiap lagu di UTS-2 agar bisa
  dinilai isi/inspirasinya;
\item
  \textbf{Tulis 1 review personal penuh} (UTS-5) memakai rubrik (pilih
  salah satu karya Anda sendiri sebagai objek);
\item
  Revisi UTS-1 dengan pembuka anekdot + refleksi nilai diri.
\end{enumerate}

Jika Anda mau, saya bisa bantu merapikan UTS-4 (tabel SHAPE + narasi)
dan membuat kerangka cepat untuk \textbf{review} di UTS-5 dari salah
satu karya Anda.

\bookmarksetup{startatroot}

\chapter{UAS-1 My Concepts}\label{uas-1-my-concepts}

Mau hidup epik ? \href{lifestory.pdf}{Write your Life Story}

Apa itu berkonsep?

\url{https://youtu.be/QVfUlVBO80U?si=yM6q_rwV9rcDBbu7}

\bookmarksetup{startatroot}

\chapter{UAS-3 My Opinions}\label{uas-3-my-opinions}

SApa itu beropini? \href{BM\%20Opini.mp4}{Opini Berpengaruh}

Bagiamana menjaadi menarik? \href{./Interesting.mp4}{Menjadi Menarik}

\bookmarksetup{startatroot}

\chapter{UAS-3 My Innovations}\label{uas-3-my-innovations}

\bookmarksetup{startatroot}

\chapter{UAS-4 My Knowledge}\label{uas-4-my-knowledge}

Cara saya mengkomunikasikan sebuah pengatahuan sebagai petunjuk bagi
orang lain 1) saya tulis
\href{Rekomendasi\%20Presentasi\%20Efektif(Contoh\%20Makalah).pdf}{makalah
sebagai bahan utama} 2) lalu saya buat
\href{Contoh\%20Transkrip\%20Presentasi.pdf}{transkrip ucapan lisan} 3)
kemudian saya kembangkan
\href{Rekomendasi\%20Presentasi\%20(Contoh\%20Slides).pdf}{slide
pendukung trnsskrip} 4) lalu saya memproduksivideo audio visual
\url{https://youtu.be/ZbghfMvnPZc} \url{https://youtu.be/ZbghfMvnPZc}

\bookmarksetup{startatroot}

\chapter{UAS-5 My Professional
Reviews}\label{uas-5-my-professional-reviews}

Untuk melAkukan review, seperti pada
\href{../My_Personal_Reviews/Doc.5.Mengevaluasi-Esai-Berdasarkan-Rubrik.pdf}{pendekatan
AI}, kita membutuhkan rubrik

\bookmarksetup{startatroot}

\chapter{Summary}\label{summary}

In summary, this book has no content whatsoever.

\bookmarksetup{startatroot}

\chapter*{References}\label{references}
\addcontentsline{toc}{chapter}{References}

\markboth{References}{References}

\phantomsection\label{refs}




\end{document}
